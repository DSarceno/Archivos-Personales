% AUTOR: Diego Sarceño

% AUTOR: Diego Sarce�o

\documentclass[letterpaper, 12pt, openany]{book} % A4 paper size, default 11pt font size and oneside for equal margins


\usepackage[T1]{fontenc} % Output font encoding for international characters
\usepackage{fouriernc} % Use the New Century Schoolbook font
\usepackage[letterpaper]{geometry} %margenes 
\geometry{verbose,tmargin=2cm,bmargin=2.5cm,lmargin=2cm,rmargin=2cm, includeall}
\usepackage{amsmath,amsthm,amssymb} %modos matemáticos y  simbolos
\usepackage{latexsym,amsfonts} %simbolos matematicos
\usepackage{cancel} %hacer la linea que cancela las ecuaciones
\usepackage[spanish, es-noshorthands]{babel} %comandos en español y cambia el cuadro por la tabla
\decimalpoint %cambia las comas por puntos decimal
\usepackage[utf8]{inputenc} %caracteristicas del español
\usepackage{physics} %Simbolos fisicos
\usepackage{array} %mejores formatos de tabla
\parindent =0.5in%sangria 
\usepackage{graphicx} %graficas e imagenes
\usepackage{mathtools} 
\usepackage[framemethod=TikZ]{mdframed}%Entornos talegas
\usepackage[colorlinks = true,
			linkcolor = black,
			citecolor = black,
			urlcolor = blue]{hyperref}%formato de los links y URL's
\usepackage{multicol} %varias columnas
\usepackage{enumerate} %enumeraciones
\usepackage{pgf,tikz,pgfplots} %documentos en formato tikz
\usepackage{mathrsfs} %letras chingonas (transformada de laplace)
\usepackage{subfigure} %varias figuras seguidas
\usepackage[square,numbers]{natbib} %bibliografias
\usepackage[nottoc]{tocbibind}
\bibliographystyle{plainnat}
\usetikzlibrary{arrows, babel}
\usepackage{tabulary}
\usepackage{multirow} %ocupar varias filas en una tabla
\usepackage{fancybox} %recuadros talegas
\usepackage{float} %ubicar graficas
%\usepackage[usenames]{color}
\usepackage{stackrel}
\usepackage{calligra}
\usepackage{lipsum}
\usepackage{cite}
\usepackage{circuitikz}
\usepackage{listings} % permite el ingreso de codigo
%\usepackage{showframe}
%\usepackage{xcolor}
%\usepackage{LobsterTwo}
%%%%%%%%%%%%%%%%%%%%%%%%--formato a de doble coumna--%%%%%%%%%%%%%%%%%%
\setlength{\columnseprule}{1pt}
%%%%%%%%%%%%%%%%%%%%%%--fancyhdr--%%%%%%%%%%%%%%%%%%%%%%%%%%%%%%%%%%%%%
\usepackage{fancyhdr}%formato de pagina
\pagestyle{fancy}%colocar la pagina con el formato deseado
\fancyhead{}
%\fancyhead[LO,RE]{LO}
%\fancyhead[C]{\thesection}
\fancyhead[RO,LE]{\footnotesize{\thepage}}
\fancyfoot{}
%\fancyfoot[L]{Diego Sarceño}
%%%%%%%%%%%%%%%%%%%%%%%%%%%%%%%%%%%%%%%%%%%%%%%%%%%%%%%%%%%
%--------------------New Commands-----------------------------
\newcommand{\N}{\mathbb{N}}
\newcommand{\Z}{\mathbb{Z}}
\newcommand{\Q}{\mathbb{Q}}
\newcommand{\I}{\mathbb{I}}
\newcommand{\R}{\mathbb{R}}
\newcommand{\C}{\mathbb{C}} %Conjuntos numericos
\newcommand{\F}{\mathbb{F}} %Campo Cualquiera
\newcommand{\f}{\textit{f}} %f de funcion
\newcommand{\g}{\textit{g}} %g de funcion
\newcommand{\kernel}{\mathscr{N}} %kernel 
\newcommand{\range}{\mathcal{R}} %range
\newcommand{\lagran}{\mathcal{L}} %lagrangiano
\newcommand{\laplace}{\mathscr{L}} %transformada de laplace, mapas lineales
\newcommand{\M}{\mathcal{M}} %Matrices
\newcolumntype{E}{>{$}c<{$}} %entorno matematico en columnas de una tabla
\newcommand{\vi}{\boldsymbol{\hat{\imath}}}
\newcommand{\vj}{\boldsymbol{\hat{\jmath}}}
\newcommand{\vk}{\vu{k}}%vectores unitarios R3
\newcommand{\vr}{\hat{r}}
\newcommand{\vp}{\boldsymbol{\hat{\theta}}}
\newcommand{\vz}{\vu{z}}%vectores unitarios en cilindricas
\newcommand{\vaz}{\boldsymbol{\hat{\phi}}}%vectores unitarios en esféricas
\newcommand\numberthis{\addtocounter{equation}{1}\tag{\theequation}}
\newcommand{\LI}{\lim _{h\longrightarrow 0}}
\newcommand{\SU}{\longrightarrow \sum _{n=0} ^{\infty}}
\newcommand{\QED}{\hfill {\blacksquare}}
\newcommand{\plogo}{$\boxed{\mathscr{PL}}$}
%----------------------------------------------------------
\newcommand{\Ev}{\mathbf{E}}
\newcommand{\rv}{\mathbf{r}}
\newcommand{\ru}{\hat{\rv}}
\newcommand{\zu}{\hat{\mathbf{z}}}
%----------------------------------------------------------


\begin{document} 

% AUTOR: Diego Sarc�eño

%----------------------------------------------------------
%	TITLE PAGE
%----------------------------------------------------------

\begin{titlepage} % Suppresses headers and footers on the title page
    \newgeometry{verbose,tmargin=2cm,bmargin=2.5cm,lmargin=2cm,rmargin=2cm}
	\centering % Centre everything on the title page
	\scshape % Use small caps for all text on the title page
	\vspace*{\baselineskip} % White space at the top of the page
	
	%------------------------------------------------
	%	Title
	%------------------------------------------------
	
	\rule{\textwidth}{1.6pt}\vspace*{-\baselineskip}\vspace*{2pt} % Thick horizontal rule
	\rule{\textwidth}{0.4pt} % Thin horizontal rule
	
	\vspace{0.75\baselineskip} % Whitespace above the title
	{\LARGE Física para Olimpistas \\} % Title
	\vspace{0.75\baselineskip} % Whitespace below the titl
	\rule{\textwidth}{0.4pt}\vspace*{-\baselineskip}\vspace{3.2pt} % Thin horizontal rule
	\rule{\textwidth}{1.6pt} % Thick horizontal rule
	\vspace{2\baselineskip} % Whitespace after the title block
	
	%------------------------------------------------
	%	Subtitle
	%------------------------------------------------
	
	Una Recopilación de los Mejores Problemas de Física de las Olimpiadas más Conocidas y de los Mejores Libros % Subtitle or further description
	\vspace*{3\baselineskip} % Whitespace under the subtitle
	
	%------------------------------------------------
	%	Editor(s)
	%------------------------------------------------
	
	Editado Por \\
	\vspace{0.5\baselineskip} % Whitespace before the editors
	{\scshape\Large Diego Sarceño} % Editor list
	\vspace{0.5\baselineskip} % Whitespace below the editor list
	%\textit{The University of California \\ Berkeley} % Editor affiliation
	\begin{figure}[H]
	    \centering
	    \includegraphics[scale=0.7]{S1.jpeg}
	    \label{LogoPersonal}
	\end{figure}        % Logo
	
	\vfill % Whitespace between editor names and publisher logo
	
	%------------------------------------------------
	%	Publisher
	%------------------------------------------------
	
	\plogo % Publisher logo
	
	\vspace{0.3\baselineskip} % Whitespace under the publisher logo
	
	2017 % Publication year
	
	{\large publisher} % Publisher

\end{titlepage}


%------------------------------------------------------------------
%-------------------------table of contents------------------------
%\renewcommand{\contentsname}{Contenido}
\restoregeometry
\tableofcontents
\thispagestyle{empty}
\newpage
%------------------------------------------------------------------
%-------------------------Contenido del documento------------------
\setcounter{page}{1}
%------DEDICATORIA----------
\section*{\textbf{\textit{Sobre el Autor}}}
\chead{Sobre el Autor}

\lipsum[1-3] \\
\marginpar{Non tortor purus mollis potenti laoreet vulputate egestas arcu leo senectus, hendrerit euismod curae praesent ridiculus dapibus.}
Lorem ipsum dolor sit amet consectetur adipiscing elit dictum neque lacinia erat, consequat ultrices nisl conubia fringilla semper laoreet porta nascetur senectus, arcu sed sociosqu orci donec natoque ante lobortis eleifend morbi. Facilisi aptent penatibus ac maecenas accumsan natoque dapibus pellentesque suscipit hac, imperdiet molestie dui bibendum tellus risus condimentum quis nibh morbi, erat in viverra ante porta platea cras orci conubia. Id magnis integer nostra dui commodo ac morbi at, dictum eros mi mattis ullamcorper leo purus mollis lectus, tempus vivamus platea fermentum pellentesque hac dapibus.
\newpage
\section*{\textit{Prefacio}}
\chead{Prefacio}
\lipsum[1-6]
\chapter{Rincón Matemático}
    En este capítulo estudiaremos conceptos básicos acerca de áreas de la matemática como álgebra lineal funciones y cálculo diferencial e integral. No indagaremos a fondo en cada área, solo en las definiciones útiles que servirán más adelante.\\
    
    En el capítulo se estudiará la definición de función, límite, derivada e integral. Estos conceptos son útiles para entender ciertos procesos o sistemas físicos de los cuales vamos a estudiar más adelante.
    \newpage
    \section{\textit{Definiciones Útiles}}
    \chead{Definiciones Útiles}
    \lipsum[1-3]
    \section{\textit{Vectores}}
    \chead{Vectores}
    \subsection{Campos}
    Para iniciar con el estudio de los vectores requerimos saber el origen de todos sus componentes. 
\chapter{Mecánica Clásica}
        \lipsum[1-2]
        \newpage
    \section{\textit{Cinemática}}
    \chead{Cinemática}
    \lipsum[1-5]
    \section{\textit{Leyes de Newton}}
    \chead{Leyes de Newton}
    \lipsum[1-5]
    \section{\textit{Trabajo y Energía}}
    \chead{Trabajo y Energía}
    \lipsum[1-5]
    \section{\textit{Momentum Lineal}}
    \chead{Momentum Lienal}
    \lipsum[1-5]
    \section{\textit{Cinemática de Cuerpos Rígidos}}
    \chead{Cinemática de Cuerpos Rígidos}
    \lipsum[1-5]
    \section{\textit{Dinámica de Cuerpos Rígidos}}
    \chead{Dinámica de Cuerpos Rígidos}
    \lipsum[1-5]
    \section{\textit{Equilibrio de Cuerpos Rígidos}}
    \chead{Equilibrio de Cuerpos Rígidos}
    \lipsum[1-5]
    \section{\textit{Gravitación}}
    \chead{Gravitación}
    \lipsum[1-5]
    \section{\textit{Mecánica de Fluídos}}
    \chead{Mecánica de Fluídos}
    \lipsum[1-5]
    \section{\textit{Movimiento Oscilatorio}}
    \chead{Movimiento Oscilatorio}
    \lipsum[1-5]
    \section{\textit{Ondas}}
    \chead{Ondas}
    \lipsum[1-5]
\chapter{Termodinámica}
\chapter{Electromagnetismo}
\chapter{Óptica}
\chapter{Mecánica Relativista}
\chapter{Física Moderna}
\chapter{Materia}
\chapter{Varios}
\end{document}

